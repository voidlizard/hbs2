\documentclass[11pt,a4paper]{article}

\usepackage{polyglossia}
\usepackage{xltxtra}
\usepackage[margin=2cm,a4paper]{geometry}% http://ctan.org/pkg/geometry
\usepackage{pdfpages}
\usepackage{graphicx}
\usepackage[ddmmyyyy]{datetime}
\usepackage{booktabs}
\usepackage{enumitem}
\usepackage{amssymb}
\usepackage{amsmath}
\usepackage{bm}
\usepackage[nomessages]{fp}
\usepackage{caption}
\usepackage{url}
\usepackage{indentfirst}
\usepackage[parfill]{parskip}
\usepackage[ colorlinks=true
           , linkcolor=black
		   , anchorcolor=black
		   , citecolor=black
		   , filecolor=black
		   , menucolor=black
		   , runcolor=black
		   , urlcolor=blue]{hyperref}
\usepackage{tikz}
\usetikzlibrary{arrows,snakes,shapes,backgrounds,positioning,calc}
\usepackage{marvosym}
\usepackage{pifont}
\usepackage{fontspec}
\usepackage{fontawesome5}
\usepackage{listings}
\usepackage{verbatim}
\usepackage{xcolor}
\usepackage{float} % Needed for the floating environment

\setmainlanguage{russian}
\defaultfontfeatures{Ligatures=TeX,Mapping=tex-text}
\setmainfont{Liberation Serif}
\newfontfamily\cyrillicfont{Liberation Serif}[Script=Cyrillic]
\newfontfamily{\cyrillicfonttt}{Liberation Mono}[Scale=0.8]

\setlist{noitemsep}
\setlength{\intextsep}{2cm}

\newcommand{\term}[2]{\textit{#2}}
\renewcommand{\dateseparator}{.}
\renewcommand*\contentsname{Содержание}

\lstset{
  language=Haskell,
  basicstyle=\ttfamily\small,
  keywordstyle=\color{blue},
  commentstyle=\color{green},
  stringstyle=\color{red},
  % numberstyle=\tiny\color{gray},
  % numbers=left,
  % stepnumber=1,
  showstringspaces=false,
  breaklines=true,
  frame=single,
}

\newfloat{Code}{t}{myc}

\graphicspath{ {img/}}

\title{Репозиторий hbs2-git}

\begin{document}

\maketitle

\section{Идея}

hbs2-git это адаптер, позволяющий git работать с HBS2 в качестве бекенда для сохранения и получения
объектов. HBS2 это распределённый P2P CAS, позволяющий обмениваться данными, синхронизировать их,
подписываться на специфические источники при помощи механизма \term{reflog}{ссылок}.

Таким образом, hbs2-git позволяет производить децентрализованную синхронизацию репозиториев без
участия какого-то выделенного сервера/сервиса, используя вместо него множество пиров, которые
находят друг-друга при помощи механизма \term{pex}{PEX} (Peer Exchange) --- то есть находят друг
друга при помощи broadcast сообщений, DNS бутстрапа и списка известных пиров, то есть примерно  теми
же способами, что используются в прочих децентрализованных сетях.

Авторизация и аутентификация осуществляются при помощи криптографических механизмов: криптоподписи
и шифрования.

Механизмы эти работают на уровне ссылок, блоков, протоколов и можно еще дополнительно шифровать
собственный контент, если будет такое желание и необходимость.

Вместо того, что бы делать $git~push$ в remote на каком-то сервере с доменным именем, доступом и
прочее, $git~push$ происходит в \term{reflog}{ссылку/рефлог}, который поддерживается всеми пирами,
которые согласны поддерживать этот рефлог.

Узел hbs2-peer может постить \term{transaction}{транзакции} обновления ссылок и получать их, а так же
получать журнал транзакций в виде дерева Меркля.

Как было сказано выше, hbs2-git использует механизм \term{reflog}{рефлога} или же просто
\term{reflog}{ссылки} для своей работы. Рефлог представляет собой некий уникальный идентификатор,
ассоциированный с публичным ключом шифрования (фактически: публичный ключ и есть), а обладание
приватным ключом от этого публичного ключа является доказательством права записи в эту ссылку.

Ссылки могут быть разных типов, рефлог --- это условное название ссылки определенного, конкретного
вида, который реализован, как примитив в HBS2 (hbs2-peer будет точнее).

Рефлог есть список транзакций, упорядоченный по хэшам их контента, представленный в виде хэш-ссылки
на \term{ann-merkle-tree}{дерево Меркля} журнала (лога) транзакций.

Предполагается, что у рефлога есть один писатель --- владелец приватного ключа.

Никакого консенсуса с рефлогом не ассоциировано, однако, он может возникать на уровне данных
транзакций и стейта.

Например, в случае git есть порядок объектов, определяемый DAG ссылки (в смысле git. т.е в основном
это бранч). Имея порядок объектов, можно вывести и порядок операций, т.к. в логе операции
упорядочены последовательно, а сами логи имеют ссылку на коммиты, для которых тоже есть порядок.

Таким образом, даже имея несколько писателей в одну ссылку, которые, вроде бы, могут <<портить>>
ссылки git друг другу, мы можем добавить в стейт идентификатор пользователя в ключ для таблицы
ссылок, и, таким образом, каждый будет видеть только <<своё>> значение ссылки, кроме того, можно
генерировать уникальные ссылки из объединения ключа пользователя и названия бранча.

На текущий момент это не реализовано (но возможно).

Содержимое транзакций может быть произвольным, но есть специальные виды транзакций,
которые поддерживаются hbs2-peer и содержат ссылку на какое-то \term{ann-merkle-tree}{дерево
Меркля}. Смысл поддержки заключается в том, что если hbs2-peer получает транзакцию такого
типа, он заодно автоматически скачивает и дерево, на которое она ссылается.

Транзакции содержат числовое поле, которое призвано их упорядочивать, однако оно может
использоваться или не использоваться, по усмотрению. Внутри списка они упорядочиваются по хэшам, а
приложение может их упорядочивать, как ему хочется или даже несколькими способами сразу.

В нашем конкретном случае, каждая транзакция ссылается на \term{log}{лог} объектов git,
соответственно, лог содержит коммиты, каждый коммит может быть упорядочен относительно прочих,
значит, и логи могут быть упорядочены, например, по максимальной высоте (глубине?) коммита,
который в нём содержится.

Теоретически, тут могут быть краевые случаи, но поскольку мы для создания этого лога
используем функциональность git, мы считаем, что содержимое лога разумно.

Можно сказать, что порядок транзакций задаётся порядком логов, на которые они ссылаются,
который определяется порядком следования коммитов в git.

При своей работе hbs2-git формирует логи git в виде файлов, содержащих заголовки секций
и секции, в которых содержатся объекты разного типа, и помещает их в HBS2 при помощи API hbs2-peer.

Далее, полученные хэш-ссылки на деревья помещаются в транзакции, после чего транзакции подписываются
и публикуются.

И обратно, что бы получить новые объекты при $git~fetch$  hbs2-git получает текущее значение
хэш-ссылки рефлога, скачивает данные рефлога, получает список транзакций, обрабатывает те, которые
до сих пор не обработаны, получает логи, сканирует логи и импортирует объекты в репозиторий git
и обновляет стейт.

\section{Компоненты}


\begin{figure}[h!]
\centering
\begin{tikzpicture}[ every label/.style={font=\scriptsize},
                     every node/.style={font=\scriptsize},
                     handle/.style={ draw=black
                                   , circle
                                   , inner sep=2pt
                                   },
                     box/.style={ draw=black
                                , rounded corners,
                                , anchor=base
                                , font=\scriptsize
                                , minimum height=1.5cm
                                , text width=1.5cm
                                , align=center
                                },
                    db/.style={ cylinder
                              , draw
                              , fill=gray!10
                              , minimum height=1cm
                              , minimum width=1.5cm
                              , shape border rotate=90
                              , aspect=0.5
                              }
                   ]

  \node[box,minimum height=3cm] (peer) {hbs2-peer};

  \node[ box
       , minimum height=1cm
       , text width=6em
       , left=5cm of peer.north west
       , anchor = north east
       ] (git-remote-hbs2) {git-remote-hbs2};

  \node[ box
       , minimum height=1cm
       , text width=6em
       , below=1cm of git-remote-hbs2.south east, anchor = north east
       ] (git-hbs2) {git-hbs2};

  \node[ rectangle
       , draw
       , dashed
       , above = 2mm of git-remote-hbs2.north west
       , xshift = -2mm
       , text width=2.8cm
       , minimum height=3.6cm
       , anchor=north west] (tools) {};

  \node[box, minimum height=1cm, below=2cm of git-hbs2.south, anchor=north] (git) {git};

  \node[db,below=1.6cm of git.north,anchor=north] (repo) {git repo};

  \draw[->] (git.south) -- ($(repo.north) - (0,+2mm)$) ;

  \draw[->] (git.north) -- ($(tools.south west)!(git.north)!(tools.south east)$)
    node[midway,right] {CLI/PIPE};

  \node[ db
       , left=1cm of git-remote-hbs2.north west, anchor=north east
       , yshift=-1cm
       ] (state) {State};

  \draw[->] (git-remote-hbs2.west) -| ($(state.north) - (0,+2mm)$) ;
  \draw[->] (git-hbs2.west) -| (state.south);

  \node[handle,left=3cm of peer.130] (catAPI) {};
  \node[handle,below=5mm of catAPI] (putAPI) {};
  \node[handle,below=5mm of putAPI] (refLogGetAPI) {};
  \node[handle,below=5mm of refLogGetAPI] (reflogSendRaw) {};


  \draw[->] (catAPI) -- ($(peer.north west)!(catAPI)!(peer.south west)$) node[above,midway] {HTTP: Cat};

  \draw[->] (putAPI) -- ($(peer.north west)!(putAPI)!(peer.south west)$) node[above,midway] {HTTP: Put};

  \draw[->] (refLogGetAPI) -- ($(peer.north west)!(refLogGetAPI)!(peer.south west)$)
    node[above, midway] {HTTP: RefLogGet};

  \draw[->] (reflogSendRaw) -- ($(peer.north west)!(reflogSendRaw)!(peer.south west)$)
    node[above, midway] {CLI: RefLogSendRaw};

  \draw[dashed] (catAPI) -- (putAPI) -- (refLogGetAPI) -- (reflogSendRaw)
  node[circle,midway,yshift=0.7cm,inner sep=2pt] (m1) {};

  \draw[->] (git-remote-hbs2.east) -- (m1);

  \draw[->] (git-hbs2.east) -- (m1);

  \node[ db
       , below=1cm of peer.south
       ] (store) {Store};

  \draw[->] (peer.south) -- ($(store.north) - (0,+2mm)$) ;

\end{tikzpicture}
\end{figure}

\subsection*{git-remote-hbs2}

Исполняемый файл, git-remote-helper, хэлпер git для протокола hbs2://

\subsection*{git-hbs2}

Исполняемый файл, различные функции для работы с hbs2-git, например,
export для первоначальной  инициализации \term{reflog}{рефлога}, или
scan для перестройки \term{state}{стейта}.

\subsection*{git}

Процесс git

\subsection*{hbs2-peer}

Процесс hbs2-peer

\subsection*{Store}

Хранилище объектов HBS2 (меркл-деревья, блоки, ссылки, ...)

\subsection*{State}

Вычисляемый \term{state}{стейт} для \term{reflog}{рефлога} репозитория.

\section{Отсортированный по времени/высоте сегментированный лог}

\begin{figure}[h!]
\centering
  \begin{tikzpicture}[every node/.append style={font=\scriptsize}]

  \node[ rectangle split
       , rectangle split parts=3
       , draw
       , font=\scriptsize
       , text width=3cm
       , label={above:Reflog}
       ] (reflog)
  {
    Transaction~N
    \nodepart{two}...
    \nodepart{three}Transaction~1
  };


  \node[ draw
       , right = 2cm of reflog.north east, anchor=north west
       , rectangle split, rectangle split horizontal, rectangle split parts=3
       , label={below:PUSH log}
       ] (pushlog)
       { \nodepart{one}$S_1$
         \nodepart{two}...
         \nodepart{three}$S_n$
       };

  \draw[-latex] (reflog.12) -- (pushlog) node[midway,above] {merkle~tree};

\end{tikzpicture}
\end{figure}

Данный формат использует механизм \term{reflog}{reflog} hbs2, который является Merkle-деревом
списка транзакций. Каждая транзакция имеет стандартный вид, и содержит ссылку на
\term{log}{лог} (хэш \term{ann-merkle-tree}{аннотированного merkle-дерева} лога).

Каждая секция $S_n$ \term{log}{лога} имеет вид:

\begin{figure}[h!]
\centering
  \begin{tikzpicture}[every node/.append style={font=\scriptsize}]

  \node[ rectangle split,  rectangle split horizontal
       , rectangle split parts=2
       , draw
       , font=\scriptsize
       , text width=3cm
       , label={above:$S_n$}
       ] (reflog)
  {
    HEAD(64)
    \nodepart{two}DATA(LogEntrySize)
  };

  \node[ rectangle split
       , rectangle split parts=3
       , draw
       , font=\scriptsize
       , text width=3cm
       , label={above:HEAD}
       , below=1cm of reflog.south west, anchor=north west
       ] (head)
  {

    \nodepart{one}LogEntryType
    \nodepart{two}LogEntryHash
    \nodepart{three}LogEntrySize : W32
  };

    \draw[->] (reflog.186) -- ($(head.north west)!(reflog.186)!(head.north east)$);

\end{tikzpicture}
\end{figure}

Где HEAD это заголовок фиксированного размера 64~байта, DATA -- байтовая строка размера (LogEntrySize)~байт
произвольного формата, зависящего от LogEntryType.

Формат HEAD:  сериализованное в CBOR значение вида $(Type, GitHash?, Size)$, где GitHash --
опционален (соответствует типу $Maybe GitHash$).

Формат DATA: сжатая GZip произвольная строка, размером LogEntrySize \textbf{в сжатом виде}.

Таким образом, данный формат может быть расширен добавлением новых конструкторов в тип
$LogEntryType$.

Текущие основные типы секций:

\begin{description}
 \item[GitLogEntryCommit] объект git типа Commit
 \item[GitLogEntryTree]   объект git типа Tree
 \item[GitLogEntryBlob]   объект git типа Blob
 \item[GitLogEntryHead]   объект типа RepoHead, сериализованный в CBOR
\end{description}

\textbf{RepoHead}

\begin{lstlisting}
data RepoHead =
  RepoHead
  { _repoHEAD  :: Maybe GitRef
  , _repoHeads :: HashMap GitRef GitHash
  }
\end{lstlisting}


\section{Операции}

\subsection{EXPORT}

\begin{enumerate}
  \item Перечислить все объекты git в порядке следования коммитов, так, что после каждого коммита
           идут его непосредственные зависимости (tree, blob)
  \item Писать данные объекты в \term{log}{лог}
  \item Писать лог в hbs2 как \term{ann-merkle-tree}{аннотированное дерево}
  \item Порождать \term{reflog-tran}{транзакцию} для \term{reflog}{рефлога} со ссылкой на дерево из
    предыдущего пункта
\end{enumerate}

\subsection{PUSH}

\begin{enumerate}
  \item Перечислить все объекты git в порядке следования коммитов начиная с последнего известного
    для данной ветки, аналогично EXPORT.
  \item Далее аналогично 2--4 EXPORT
\end{enumerate}

\subsection{FETCH}

\begin{enumerate}
  \item Получить \term{hashref}{значение HashRef} для \term{reflog}{рефлога}
  \item Обойти все необработанные \term{transaction}{транзакции} \term{reflog}{рефлога}
  \item Для каждой такой транзакции извлечь ссылку на \term{merkle-tree}{дерево}
  \item Получить \term{log}{лог} для этого \term{merkle-tree}{дерева}
  \item Обойти секции лога, импортировать объекты в репозиторий git, устанавливать
        ссылки (в \term{state}{стейте})
  \item Обновлять \term{state}{стейт}
\end{enumerate}

Поскольку одна операция $git~push$ устанавливает одну ссылку в заданное значение коммита,
у нас есть соответствие $refname \leftrightarrow commit$ для каждого лога.

Поскольку для коммитов определен порядок, мы можем вычислить порядок логов в журнале транзакций
и порядок следования операций установления ссылки.
к
Таким образом, мы можем вычислить актуальные (последние) значения для каждой ссылки для
объединения логов.

\subsection{DELETE}

Удалять ссылку. Удаление ссылки производится путем установки её в значение
'0\{40\}' (хэш из сорока нулей) а не в пустую строку, как это происхоит в git,
что бы не рассматривать частный случай при разборе объектов типа RepoHead.

В \term{state}{стейте} происходит фильтрация и ссылки, установленные в
данное значение --- не выводятся.

\section{Стейт}

Каждому \term{reflog}{рефлогу} соответствует  \term{state}{стейт}, который вычисляется и
синхронизируется при операциях FETCH и PUSH.

\term{state}{Стейт} представляет собой БД SQLite которая содержит необходимую для функционирования
hbs2-git информацию (объекты, отношения, логи, индексы).

\term{state}{Стейт} может быть в любой момент пересоздан из рефлога.

Операции из рефлога над \term{state}{стейтом} идемпотентны.

Порядок существенных операций не играет роли, однако стейт содержит временные метки,
которые будут меняться в зависимости от времени.

То есть, существенная часть стейта определяется только операциями из рефлога, причем,
порядок неважен, так как значимые сущности имеют порядок, определяемый репозиторием git
(топологический).

\section{Замечания}

\subsection*{Перечисление объектов репозитория}

Перечисление объектов репозитория осуществляется при помощи команды

\begin{verbatim}
git rev-list --objects --in-commit-order --reverse --date-order
\end{verbatim}

и введение явной сортировки приводит к тому, что на больших репозиториях
данная команда сначала перечисляет все объекты, и потом их сортирует, что
происходит очень долго.

Явно заданный порядок нужен для возможности повторного использования
частей логов для уменьшения оверхеда по данным, однако эффективность
данной меры пока не оценена, а замедление на больших репозиториях очень
значительное.

Цель частичного повторного использования сегментов лога может быть
достигнута, если одному \term{PUSH}{PUSH} или \term{EXPORT}{EXPORT} будет
соответствовать не один лог, а некоторое количество логов, с ограничением
каждого или по размеру, или по количеству объектов.

Это не создаёт никаких дополнительных расходов, следовательно, должно
быть реализовано.

Тогда существует вероятность, что некоторые последовательности коммитов
будут повторно использоваться в системе, и кажется, это лучше, чем текущая
ситуация с попыткой переиспользования частей лога.

Откуда берётся переиспользование вообще: так как лог сохраняется, как
\term{merkle}{дерево Меркля}, он разбивается на сегменты фиксированного
размера ($\approx{}256K$) и эти сегменты могут не скачиваться повторно, если они
уже есть, или же скачиваться у разных участников сети, если у них они
есть.

\subsection*{Использование fast-export/fast-import}

Хотя чтение и импорт объектов при помощи fast-export/fast-import
может быть быстрее, а сжатие такого лога полностью при помощи
gzip --- лучше, надо заметить следующее:

\begin{enumerate}
  \item Отсутствие хэшей в этом формате,
        которые заменены короткими уникальными идентификаторами
  \item Очень сильное замедление сжатия с ростом объема лога fast-export
  \item Быстрое чтение объектов при помощи fast-export
  \item Вероятно, более быстрый импорт объектов при помощи fast-import
\end{enumerate}

\subsubsection*{Следствия}

На больших логах будет работать очень медленно, так же медленно, как и сейчас.

На маленьких логах всё работает достаточно быстро и сейчас, нет смысла утруждаться.

Хэши нужны, что бы контролировать целостность логов, то есть придется парсить формат fast-import
и их считать, что еще больше замедлит.

Если хэши объектов не считать, то есть опасность постинга по сути мусора в репозиторий git, т.к.
нет возможности контролировать и сопоставлять объекты.

Текущий формат лога позволяет добавить секции, где будут логи формата fast-import.

Это может привести к некоторому ускорению, если у нас не будет огромных логов, а логи будут нарезаны
по некоему разумному размеру.

Выглядит так, что можно попробовать сделать это в дальнейшем, когда другие способы оптимизации
скорости будут исчерпаны.

\end{document}


